\documentclass{beamer}

%\usepackage{lipsum}             % for dummy text
%\usepackage{stix}               % use the STIX font (of course you can delete this line)
\usepackage[BoldFont,SlantFont]{xeCJK}

%<Blackboard>
\usetheme{Blackboard}
%<DarkConsole>\usetheme{DarkConsole}
%<LightConsole>\usetheme{LightConsole}
%<Notebook>\usetheme{Notebook}

\title{An Example of \texttt{kmbeamer}}
%<Blackboard>\subtitle{Blackboard theme}
%<DarkConsole>\subtitle{DarkConsole theme}
%<LightConsole>\subtitle{LightConsole theme}
%<Notebook>\subtitle{Notebook theme}
\author{李传玺\footnote{\texttt{molakirlee@163.com}}}

\begin{document}

\begin{frame}
  \maketitle
\end{frame}

\begin{frame}{Outline}
  \tableofcontents
\end{frame}

\section{Mathematical Story}

\begin{frame}{Slide 1}
  This is a very mathematical sentence.

  \pause

  The followings are mathematical lists.

  \begin{enumerate}
  \item Item $1$\pause
  \item Item $1+1$\pause
  \item Item $1+1+1$
  \end{enumerate}

  \pause

  \begin{itemize}
  \item Item $1+1+1+1$\pause
  \item Item $1+1+1+1+1$\pause
  \item Item $1+1+1+1+1+1$
  \end{itemize}
\end{frame}


\begin{frame}{Slide 1+1}
  \alert{Get started in writing equations!!!}

  \begin{theorem}[Gaussian integral]
    The following integral is very well known:
    \begin{equation}
      \int_{-\infty}^\infty \mathrm{e}^{-x^2}\,\mathrm{d}x=\sqrt{\pi}.
    \end{equation}

  \end{theorem}
\end{frame}


\section{More Mathematical Story}


\end{document}